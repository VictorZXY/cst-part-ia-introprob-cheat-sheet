\documentclass{article}
\usepackage[a4paper, left=25.4mm, top=25.4mm, right=25.4mm, bottom=25.4mm]{geometry}
\usepackage[dvipsnames]{xcolor}
\usepackage[hidelinks]{hyperref}
\usepackage[shortlabels]{enumitem}
\usepackage{amsmath}
\usepackage{amssymb}
\usepackage{amsthm}
\usepackage{authoraftertitle}
\usepackage{blindtext}
\usepackage{booktabs}
\usepackage{bussproofs}
\usepackage{bm}
\usepackage{color}
\usepackage{fancyhdr}
\usepackage{graphicx}
\usepackage{latexsym}
\usepackage{multicol}
\usepackage{newpxtext}
%\usepackage{newpxmath}
\usepackage{ragged2e}
\usepackage{subcaption}
\usepackage{textcomp}
\usepackage{textgreek}
\usepackage{vwcol}
\renewcommand{\headrulewidth}{0pt}
\renewcommand{\qedsymbol}{Q.E.D.}
\renewcommand{\sfdefault}{lmss}
\renewcommand{\thesubsection}{\arabic{subsection}}

\author{Victor Zhao\\xz398@cantab.ac.uk}

\begin{document}
\centering
\section*{Introduction to Probability\\CST Part IA Paper 1}
\MyAuthor

\justifying

\subsection{Prerequisites and Introduction}

\begin{enumerate}
	\item Combinatorics:
		
		\renewcommand\arraystretch{1.5}
		\begin{tabular}{|c|c|c|c|}
			\hline
			\multicolumn{4}{|c|}{Counting tasks on $n$ objects} \\
			\hline
			\multicolumn{2}{|c|}{Permutations (sort objects)} & \multicolumn{2}{|c|}{Combinations (choose $r$ objects)} \\
			\hline
			Distinct & Indistinct & Distinct 1 group & Distinct $k$ groups \\
			\hline
			$n!$ & $\dfrac{n!}{n_1!n_2!\cdots n_r!}$ & $\displaystyle{n\choose r}=\dfrac{n!}{r!(n-r)!}$ & $\displaystyle{n\choose n_1,n_2,\cdots,n_k}=\dfrac{n!}{n_1!n_2!\cdots n_k!}$ \\
			\hline
		\end{tabular}
	
		Pascal's identity: $\displaystyle{n\choose r} = {n-1\choose r-1}+{n-1\choose r}\qquad(1\leq r\leq n)$
		
		Binomial theorem: $(x+y)^n = \displaystyle\sum_{r=0}^{n}{n\choose r}x^r y^{n-r}$
		
	\item Frequentist definition of probability: $\mathbb{P}[E]=\displaystyle\lim_{n\to\infty}\frac{\text{\# Trials where $E$ occurs}}{\text{\# Total trials ($n$)}}$
	\item Probability axioms:
	
		Axiom 1: For any event $E$, $0\leq\mathbb{P}[E]\leq1$ \\
		Axiom 2: Probability of the sample space $S$ is $\mathbb{P}[S]=1$ \\
		Axiom 3: If $E$ and $F$ are mutually exclusive (i.e., $E\cap F=\varnothing$), then $\mathbb{P}[E\cup F]=\mathbb{P}[E]+\mathbb{P}[F]$ \\
		\phantom{Axiom 3: }In general, for all mutually exclusive events $E_1$, $E_2$, $\cdots$,
		\vspace*{-0.5\baselineskip}$$\displaystyle\mathbb{P}\left[\bigcup_{i=1}^\infty E_i\right]=\sum_{i=1}^\infty \mathbb{P}[E_i]$$
		
	\item General inclusion-exclusion principle: $\displaystyle\mathbb{P}\left[\bigcup_{i=1}^n E_i\right]=\sum_{r=1}^{n}(-1)^{r+1}\left(\sum_{i_1<\cdots<i_r}^{n}\mathbb{P}[E_{i_1}\cap\cdots\cap E_{i_r}]\right)$ 
		
		Case $n=2$: $\mathbb{P}[E\cup F]=\mathbb{P}[E]+\mathbb{P}[F]-\mathbb{P}[E\cap F]$ \\
	
	\item Union bound (Boole's inequality): For any events $E_1$, $E_2$, $\cdots$, $E_n$,
		\vspace*{-0.5\baselineskip}$$\displaystyle\mathbb{P}\left[\bigcup_{i=1}^n E_i\right]\leq\sum_{i=1}^{n}\mathbb{P}[E_i]$$
	
\end{enumerate}

\subsection{Random Variables}

\begin{enumerate}
	\item 
\end{enumerate}

\subsection{Moments and Limit Theorems}

\begin{enumerate}
	\item 
\end{enumerate}

\subsection{Applications and Statistics}

\begin{enumerate}
	\item 
\end{enumerate}

\end{document}
