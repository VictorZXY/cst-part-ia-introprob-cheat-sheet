\documentclass{article}
\usepackage[a4paper, left=25.4mm, top=25.4mm, right=25.4mm, bottom=25.4mm]{geometry}
\usepackage[dvipsnames]{xcolor}
\usepackage[hidelinks]{hyperref}
\usepackage[shortlabels]{enumitem}
\usepackage{amsmath}
\usepackage{amssymb}
\usepackage{amsthm}
\usepackage{authoraftertitle}
\usepackage{blindtext}
\usepackage{booktabs}
\usepackage{bussproofs}
\usepackage{bm}
\usepackage{color}
\usepackage{fancyhdr}
\usepackage{graphicx}
\usepackage{latexsym}
\usepackage{makecell}
\usepackage{multicol}
\usepackage{multirow}
\usepackage{newpxtext}
%\usepackage{newpxmath}
\usepackage{nicefrac}
\usepackage{ragged2e}
\usepackage{subcaption}
\usepackage{textcomp}
\usepackage{textgreek}
\usepackage{vwcol}
\renewcommand{\headrulewidth}{0pt}
\renewcommand{\qedsymbol}{Q.E.D.}
\renewcommand{\sfdefault}{lmss}
\renewcommand{\thesubsection}{\arabic{subsection}}

\author{Victor Zhao\\xz398@cantab.ac.uk}

\begin{document}
\centering
\section*{Introduction to Probability\\CST Part IA Paper 1}
\MyAuthor

\justifying

\subsection{Prerequisites and Introduction}

\begin{enumerate}
	\item Combinatorics:
	
		\renewcommand\arraystretch{1.5}
		\begin{tabular}{|c|c|c|c|}
			\hline
			\multicolumn{4}{|c|}{Counting tasks on $n$ objects} \\
			\hline
			\multicolumn{2}{|c|}{Permutations (sort objects)} & \multicolumn{2}{|c|}{Combinations (choose $r$ objects)} \\
			\hline
			Distinct & Indistinct & Distinct 1 group & Distinct $k$ groups \\
			\hline
			$n!$ & $\dfrac{n!}{n_1!n_2!\cdots n_r!}$ & $\displaystyle{n\choose r}=\dfrac{n!}{r!(n-r)!}$ & $\displaystyle{n\choose n_1,n_2,\cdots,n_k}=\dfrac{n!}{n_1!n_2!\cdots n_k!}$ \\
			\hline
		\end{tabular}
	
		Pascal's identity: $\displaystyle{n\choose r} = {n-1\choose r-1}+{n-1\choose r}\qquad(1\leq r\leq n)$
		
		Binomial theorem: $(x+y)^n = \displaystyle\sum_{r=0}^{n}{n\choose r}x^r y^{n-r}$
		
%	\item Frequentist definition of probability: $\mathbb{P}[E]=\displaystyle\lim_{n\to\infty}\frac{\text{\# Trials where $E$ occurs}}{\text{\# Total trials ($n$)}}$

	\item Probability axioms:
	
		Axiom 1: For any event $E$, $0\leq\mathbb{P}[E]\leq1$
		
		Axiom 2: Probability of the sample space $S$ is $\mathbb{P}[S]=1$
		
		Axiom 3: If $E$ and $F$ are mutually exclusive (i.e., $E\cap F=\varnothing$), then $\mathbb{P}[E\cup F]=\mathbb{P}[E]+\mathbb{P}[F]$ \\
		\phantom{Axiom 3: }In general, for all mutually exclusive events $E_1$, $E_2$, $\cdots$,
		$$\displaystyle\mathbb{P}\left[\bigcup_{i=1}^\infty E_i\right]=\sum_{i=1}^\infty \mathbb{P}[E_i]$$
		
	\item General inclusion-exclusion principle: $\displaystyle\mathbb{P}\left[\bigcup_{i=1}^n E_i\right]=\sum_{r=1}^{n}(-1)^{r+1}\left(\sum_{i_1<\cdots<i_r}^{n}\mathbb{P}[E_{i_1}\cap\cdots\cap E_{i_r}]\right)$ 
		
		Case $n=2$: $\mathbb{P}[E\cup F]=\mathbb{P}[E]+\mathbb{P}[F]-\mathbb{P}[E\cap F]$
	
	\item Union bound (Boole's inequality): For any events $E_1$, $E_2$, $\cdots$, $E_n$,
		$$\displaystyle\mathbb{P}\left[\bigcup_{i=1}^n E_i\right]\leq\sum_{i=1}^{n}\mathbb{P}[E_i]$$
		
	\item Conditional probability (original and conditioning on event \textcolor{blue}{$G$}):
		\vspace*{-\baselineskip}
		\begin{align*}
			\intertext{Chain rule:}
			&\mathbb{P}[EF]=\mathbb{P}[E|F]\mathbb{P}[F] &&\hspace*{-6cm}\mathbb{P}[EF|\textcolor{blue}{G}]=\mathbb{P}[E|F\textcolor{blue}{G}]\mathbb{P}[F|\textcolor{blue}{G}] \\
			\intertext{Multiplication rule:}
			&\mathbb{P}[E_1E_2\cdots E_n]=\mathbb{P}[E_1]\mathbb{P}[E_2|E_1]\cdots\mathbb[E_n|E_1\cdots E_{n-1}] && \\
			&\mathbb{P}[E_1E_2\cdots E_n|\textcolor{blue}{G}]=\mathbb{P}[E_1|\textcolor{blue}{G}]\mathbb{P}[E_2|E_1\textcolor{blue}{G}]\cdots\mathbb[E_n|E_1\cdots E_{n-1}\textcolor{blue}{G}] && \\
			\intertext{Independence of $E$ and $F$:}
			&\mathbb{P}[EF]=\mathbb{P}[E]\mathbb{P}[F] &&\hspace*{-6cm}\mathbb{P}[EF|\textcolor{blue}{G}]=\mathbb{P}[E|\textcolor{blue}{G}]\mathbb{P}[F|\textcolor{blue}{G}] \\
			&\mathbb{P}[E|F]=\mathbb{P}[E] &&\hspace*{-6cm}\mathbb{P}[E|F\textcolor{blue}{G}]=\mathbb{P}[E|\textcolor{blue}{G}] \\
			\intertext{Law of total probability:}
			&\mathbb{P}[E]=\mathbb{P}[EF]+\mathbb{P}[EF^\complement]=\mathbb{P}[E|F]\mathbb{P}[F]+\mathbb{P}[E|F^\complement]\mathbb{P}[F^\complement] && \\
			&\mathbb{P}[E|\textcolor{blue}{G}]=\mathbb{P}[EF|\textcolor{blue}{G}]+\mathbb{P}[EF^\complement|\textcolor{blue}{G}]=\mathbb{P}[E|F\textcolor{blue}{G}]\mathbb{P}[F|\textcolor{blue}{G}]+\mathbb{P}[E|F^\complement\textcolor{blue}{G}]\mathbb{P}[F^\complement|\textcolor{blue}{G}] && \\
			\intertext{In general, for disjoint events $F_1$, $F_2$, $\cdots$, $F_n$ such that $F_1\cup\cdots\cup F_n=S$,}
			&\mathbb{P}[E]=\sum_{i=1}^{n}\mathbb{P}[E|F_i]\mathbb{P}[F_i] &&\hspace*{-6cm}\mathbb{P}[E|\textcolor{blue}{G}]=\sum_{i=1}^{n}\mathbb{P}[E|F_i\textcolor{blue}{G}]\mathbb{P}[F_i|\textcolor{blue}{G}] \\
			\intertext{Bayes' theorem:}
			&\mathbb{P}[F|E]=\frac{\mathbb{P}[E|F]\mathbb{P}[F]}{\mathbb{P}[E]} &&\hspace*{-6cm}\mathbb{P}[F|E\textcolor{blue}{G}]=\frac{\mathbb{P}[E|F\textcolor{blue}{G}]\mathbb{P}[F|\textcolor{blue}{G}]}{\mathbb{P}[E|\textcolor{blue}{G}]}
		\end{align*}
	
	\item Confusion matrix:
	
		\begin{tabular}{|c|l|l|l|}
			\cline{3-4}
			\multicolumn{2}{c|}{} & \multicolumn{2}{c|}{Actual condition} \\
			\cline{2-4}
			\multicolumn{1}{c|}{} & Total population & Positive $F$ & Negative $F^\complement$ \\
			\hline
			\multirow{2}{*}{\makecell[l]{{}\\Predicted\\condition}} & Positive $E$ & \makecell[l]{True positive\\$\mathbb{P}[E|F]$} & \makecell[l]{False positive\\$\mathbb{P}[E|F^\complement]$} \\
			\cline{2-4}
			& Negative $E^\complement$ & \makecell[l]{False negative\\$\mathbb{P}[E^\complement|F]$} & \makecell[l]{True negative\\$\mathbb{P}[E^\complement|F^\complement]$} \\
			\hline
		\end{tabular}
\end{enumerate}

\subsection{Random Variables}

\begin{enumerate}
	\item Probability distribution functions:
		\vspace*{-0.5\baselineskip}
		\begin{multicols}{2}
			Discrete random variable $X$:
			\begin{itemize}
				\item Probability mass function (PMF): $p(x)$
				\item Compute probability:
					\begin{align*}
						&\mathbb{P}[X=a]=p(x) \\
						&\mathbb{P}[a\leq X\leq b]=\sum_{x=a}^{b}p(x)
					\end{align*}
				\item Cumulative distribution function (CDF):
					$$F_X(a)=\mathbb{P}[X\leq a]=\sum_{x\leq a}p(x)$$
			\end{itemize}
			
			\columnbreak
			Continuous random variable $X$:
			\begin{itemize}
				\item Probability density function (PDF): $f(x)$
				\item Compute probability:
					\begin{align*}
						&\mathbb{P}[X=a]=0 \\
						&\mathbb{P}[a\leq X\leq b]=\int_{a}^{b}f(x)dx
					\end{align*}
				\item Cumulative distribution function (CDF):
					$$F_X(a)=\mathbb{P}[X\leq a]=\int_{-\infty}^{a}f(x)dx$$
			\end{itemize}
		\end{multicols}
		
	\vspace*{-\baselineskip}
	\item Expectation:
		\vspace*{-0.5\baselineskip}
		\begin{multicols}{2}
			Discrete random variable $X$:
			\begin{align*}
				&\mathbb{E}[X]=\sum_x xp(x) \\
				&\mathbb{E}[g(X)]=\sum_x g(x)p(x)
			\end{align*}
			
			\columnbreak
			Continuous random variable $X$:
			\begin{align*}
				&\mathbb{E}[X]=\int_{-\infty}^{\infty} xf(x)dx \\
				&\mathbb{E}[g(X)]=\int_{-\infty}^{\infty} g(x)f(x)dx
			\end{align*}
		\end{multicols}
		\vspace*{-0.5\baselineskip}
		Linearity of expectation: $\mathbb{E}[aX+b]=a\mathbb{E}[X]+b$
		
		Additivity of expectation: $\mathbb{E}[X+Y]=\mathbb{E}[X]+\mathbb{E}[Y]$
	
	\item Variance: $\mathbb{V}[X]=\mathbb{E}\left[(X-\mathbb{E}[X])^2\right]=\mathbb{E}\left[X^2\right]-\mathbb{E}[X]^2$
	
		Scaling of variance: $\mathbb{V}[aX+b]=a^2\mathbb{V}[X]$
		
		Standard deviation: $\mathbb{SD}[X]=\sqrt{\mathbb{V}[X]}$
		
		Scaling of standard deviation: $\mathbb{SD}[aX+b]=|a|\mathbb{SD}[X]$
		
	\item Discrete distributions:
		\vspace*{-\baselineskip}
		\begin{align*}
			\intertext{Bernoulli $\mathrm{Ber}(p)$: 1 experiment with success probability $p$}
			&\mathbb{P}[X=1]=p &&\mathbb{E}[X]=p &&\mathbb{V}[X]=p(1-p) \\
			\intertext{Binomial $\mathrm{Bin}(n,p)$: $n$ independent trials with success probability $p$}
			&\mathbb{P}[X=k]={n\choose k}p^k(1-p)^{n-k} &&\mathbb{E}[X]=np &&\mathbb{V}[X]=np(1-p) \\
			\intertext{Poisson $\mathrm{Pois}(\lambda)$: \# successes over experiment duration, with success rate $\lambda=np$}
			&\mathbb{P}[X=k]=\frac{\lambda^k}{k!}e^{-\lambda} &&\mathbb{E}[X]=\lambda &&\mathbb{V}[X]=\lambda \\
			\intertext{Geometric $\mathrm{Geo}(p)$: \# independent trials until first success, with success probability $p$}
			&\mathbb{P}[X=n]=(1-p)^{n-1}p &&\mathbb{E}[X]=\frac{1}{p} &&\mathbb{V}[X]=\frac{1-p}{p^2} \\
			\intertext{Negative binomial $\mathrm{NegBin}(r,p)$: \# independent trials until $r$ success, with success probability $p$}
			&\mathbb{P}[X=n]={n-1\choose r-1}(1-p)^{n-r}p^r &&\mathbb{E}[X]=\frac{r}{p} &&\mathbb{V}[X]=\frac{r(1-p)}{p^2} \\
			\intertext{Hypergeometric $\mathrm{Hyp}(N,n,m)$: \# objects with a feature in a sample of size $n$ (without replacement) from a population of size $N$ that contains $m$ items with the feature}
			&\mathbb{P}[X=n]=\frac{{m\choose i}{N-m\choose n-i}}{{N\choose n}} &&\mathbb{E}[X]=n\frac{m}{N} &&\mathbb{V}[X]=n\frac{m}{N}\left(1-\frac{m}{N}\right)\left(1-\frac{n-1}{N-1}\right)
		\end{align*}
		
	\item Continuous distributions:
		\vspace*{-\baselineskip}
		\begin{align*}
			\intertext{Uniform $\mathrm{Uni}(\alpha,\beta)$: equal probability within range $[\alpha,\beta]$\vspace*{-0.5\baselineskip}}
			&\text{PDF: }f(x)=\begin{cases}
				\frac{1}{\beta-\alpha} &\text{when $\alpha\leq x\leq\beta$} \\
				0 &\text{otherwise}
			\end{cases} &&\hspace*{-3.9cm}\text{CDF: }F(x)=\begin{cases}
				0 &\text{when $x<\alpha$} \\
				\frac{x-\alpha}{\beta-\alpha} &\text{when $\alpha\leq x\leq\beta$} \\
				1 &\text{when $x>\beta$}
			\end{cases} \\
			&\mathbb{E}[X]=\frac{\alpha+\beta}{2} &&\hspace*{-3.9cm}\mathbb{V}[X]=\frac{(\beta-\alpha)^2}{12} \\
			\intertext{Exponential $\mathrm{Exp}(\lambda)$: time until first success occurs, with success rate $\lambda$}
			&\text{PDF: }f(x)=\begin{cases}
				\lambda e^{-\lambda x} &\text{when $x\geq0$} \\
				0 &\text{otherwise}
			\end{cases} &&\hspace*{-3.9cm}\text{CDF: }F(x)=1-e^{-\lambda x} \\
			&\mathbb{E}[X]=\frac{1}{\lambda} &&\hspace*{-3.9cm}\mathbb{V}[X]=\frac{1}{\lambda^2} \\
			\intertext{Normal (Gaussian) $\mathcal{N}(\mu,\sigma^2)$: mean $\mu$, variance $\sigma^2$}
			&\text{PDF: }f(x)=\frac{1}{\sigma\sqrt{2\pi}}e^{-\frac{(x-\mu^2)}{2\sigma^2}}&&\hspace*{-3.9cm}\text{CDF: }F(x)=\Phi\left(\frac{x-\mu}{\sigma}\right) \\
			&\mathbb{E}[X]=\mu &&\hspace*{-3.9cm}\mathbb{V}[X]=\sigma^2 \\[0.25\baselineskip]
			&X\sim\mathcal{N}(\mu,\sigma^2)\implies aX+b\sim\mathcal{N}(a\mu+b,a^2\sigma^2) \\[0.25\baselineskip]
			&X\sim\mathcal{N}(\mu_X,\sigma_X^2),Y\sim\mathcal{N}(\mu_Y,\sigma_Y^2)\implies X+Y\sim\mathcal{N}(\mu_X+\mu_Y,\sigma_X^2+\sigma_Y^2)
		\end{align*}
	
	\item Continuity correction:
		\vspace*{-0.5\baselineskip}
		\begin{align*}
			&\text{Discrete} &&        &&\text{Continuous}\hspace*{8cm} \\
			&\mathbb{P}[X=a] &&\approx &&\mathbb{P}[a-0.5\leq X\leq a+0.5] \\
			&\mathbb{P}[X>a] &&\approx &&\mathbb{P}[X\geq a+0.5] \\
			&\mathbb{P}[X\geq a] &&\approx &&\mathbb{P}[X\geq a-0.5] \\
			&\mathbb{P}[X<a] &&\approx &&\mathbb{P}[X\leq a-0.5] \\
			&\mathbb{P}[X\leq a] &&\approx &&\mathbb{P}[X\leq a+0.5]
		\end{align*}	
		
	\item Joint probability mass function (for discrete RVs): $p_{X,Y}(a,b)=\mathbb{P}[X=a,Y=b]$
	
		Joint distribution function (for discrete or continuous RVs): $F_{X,Y}(a,b)=\mathbb{P}[X\leq a,Y\leq b]$
		
		Joint probability density $f$ and joint continuous distribution $F$ (for continuous RVs):
		\begin{align*}
			&F(a,b)=\int_{-\infty}^{a}\int_{-\infty}^{b}f(x,y)dxdy &&\hspace*{-4.4cm}f(x,y)=\frac{\partial^2}{\partial x\partial y}F(x,y) \\
			&\mathbb{P}[a_1\leq X\leq b_1, a_2\leq Y\leq b_2]=\int_{a_1}^{b_1}\int_{a_2}^{b_2}f(x,y)dxdy
		\end{align*}
		
		Marginal distribution: $F_X(a)=\mathbb{P}[X\leq a]=\displaystyle\lim_{b\to\infty}F_{X,Y}(a,b)$
	
	\item Covariance: $\mathrm{Cov}[X,Y]=\mathbb{E}\left[(X-\mathbb{E}[X])(Y-\mathbb{E}[Y])\right]=\mathbb{E}[XY]-\mathbb{E}[X]\mathbb{E}[Y]$
		
		Covariance of linear combinations:
		\begin{align*}
			&\mathrm{Cov}[X,a]=0 &&\mathrm{Cov}[X,X]=\mathbb{V}[X] \\
			&\mathrm{Cov}[aX,bY]=ab\mathrm{Cov}[X,Y] &&\mathrm{Cov}[X+a,Y+b]=\mathrm{Cov}[X,Y]
		\end{align*}
		
		Variance of sum: $\mathbb{V}[X+Y]=\mathbb{V}[X]+\mathbb{V}[Y]+2\mathrm{Cov}[X,Y]$
		
		In general, for any random variables $X_1$, $X_2$, $\cdots$, $X_n$:
		$$\mathbb{V}\left[\sum_{i=1}^{n}X_i\right]=\sum_{i=1}^{n}\mathbb{V}[X_i]+2\sum_{i=1}^{n-1}\sum_{j=i+1}
		^{n}\mathrm{Cov}[X_i,X_j]$$
		
		Correlation coefficient: $\rho(X,Y)=\dfrac{\mathrm{Cov}[X,Y]}{\sqrt{\mathbb{V}[X]\mathbb{V}[Y]}}\in[-1,1]$\ \ ($\rho(X,Y)=0$ if $\mathbb{V}[X]=0$ or $\mathbb{V}[Y]=0$)
		
		Scaling-invariance of correlation coefficient: $\rho(aX,bY)=\rho(X,Y)$
	
\end{enumerate}

\subsection{Moments and Limit Theorems}

\begin{enumerate}
	\item Markov's inequality: for any non-negative random variable $X$ with finite $\mathbb{E}[X]$, for any $a>0$,
		$$\mathbb{P}[X\geq a]\leq\frac{\mathbb{E}[X]}{a}$$
		Let $a=\delta\cdot\mathbb{E}[X]$ (where $\delta>0$), then the inequality can be rewritten as
		$$\mathbb{P}[X\geq\delta\cdot\mathbb{E}[X]]\leq\frac{1}{\delta}$$
		
	\item Chebyshev's inequality: for any random variable $X$ with finite $\mathbb{E}[X]$ and $\mathbb{V}[X]$, for any $a>0$,
		$$\mathbb{P}[|X-\mathbb{E}[X]|\geq a]\leq\frac{\mathbb{V}[X]}{a^2}$$
		Let $a=\sqrt{\delta\cdot\mathbb{V}[X]}$ (where $\delta>0$), then the inequality can be rewritten as
		$$\mathbb{P}\left[|X-\mathbb{E}[X]|\geq\sqrt{\delta\cdot\mathbb{V}[X]}\right]\leq\frac{1}{\delta}$$
	
	\item Weak law of large numbers: let $\overline{X}_n=\frac{1}{n}\sum_{i=1}^{n}X_i$, where $X_i$'s are independent and identically distributed (i.i.d.) with finite expectation $\mu$ and finite variance $\sigma^2$. Then, for any $\epsilon>0$,
		$$\lim_{n\to\infty}\mathbb{P}\left[|\overline{X}_n-\mu|>\epsilon\right]=0$$
		Strong law of large numbers: 
		$$\mathbb{P}\left[\lim_{n\to\infty}\overline{X}_n=\mu\right]=1$$
		
	\item Central limit theorem: let $X_1$, $X_2$, $\cdots$, $X_n$ be any sequence of i.i.d.~random variables with finite expectation $\mu$ and finite variance $\sigma^2$. Let
		$$Z_n=\sqrt{n}\cdot\frac{\overline{X}_n-\mu}{\sigma}=\frac{1}{\sigma\sqrt{n}}\left(\sum_{i=1}^{n}X_i-n\mu\right)$$
		Then for any number $a\in\mathbb{R}$, it holds that 
		$$\lim_{n\to\infty}F_{Z_n}(a)=\Phi(a)=\frac{1}{2\pi}\int_{-\infty}^{a}e^{-\frac{x^2}{2}}dx$$
		where $\Phi$ is the CDF of the standard normal distribution $\mathcal{N}(0,1)$.
\end{enumerate}

\subsection{Applications and Statistics}

\begin{enumerate}
	\item 
\end{enumerate}

\end{document}
